%packages
\documentclass{article}
\usepackage[a4paper, total={6in, 8in}]{geometry}
\usepackage{graphicx, import} %Importing figures
\usepackage{amsmath, amsfonts, dsfont, bm} %Math typesetting
\usepackage{tikz, pgfplots, float} %For making pretty pictures
\usepackage{xcolor} %For colored text
\usepackage{enumerate} %Easy enumeration 
\usepackage{pdfpages} %Import pdf in latex file
\usepackage{physics, esint} %Useful operators
\usepackage{hyperref, cleveref} %Referencing 
 
%Settings
\pgfplotsset{compat=1.16}
\synctex=0
 
\title{Title}
\author{Names}
\date{Date}
 
\begin{document}
\maketitle

\section{Introduction}
Why do we want to even get time series, what has already been done on this topic and what method are we going to consider in this report. 

\section{Mathematical Background}
\subsection{Problem description}
Just some basic definitions and the mathematical description of the problem we want to solve. 
\subsection{Path signature}
Motivation of using the path signature to represent paths, discuss some of its properties e.g. the universal approximation. Information is in the primer and the bachelor thesis.  
\subsection{Reservoir computing}
Explain what reservoir computing is and how can we use the path signature in reservoir computing to increase the efficiency. See chapter 4 of the bachelor thesis.
\subsection{Randomized signature}
Motivation why it makes a lot more sense to use the randomized signature, see chapter 5 of the bachelor thesis. Now introduce the randomized signature as in "Universal randomised signatures for generative time series modelling". For sure we need to discuss proposition 8 (Universality of random feature neural networks) and chapter 3.

\section{Numerical experiments}
\subsection{General setup}
Explain what the code does in general terms. Input of data, the split of this data, what checks we do to verify the results, how we compute the confidence intervals. Also explain how we chose the hyper-parameters, add the final hyper-parameters to the table in the appendix. 

\subsection{Brownian motion}
\subsubsection{Setup}
Definition of Brownian motion, specify the parameters and the seed. 

\subsubsection{Results}
Show results, leave discussion for later.

\subsection{Another example?}
\subsubsection{Setup}
\subsubsection{Results}

\section{Discussion and conclusion}
Discuss the results and conclude about the method. 

\appendix
\section{Hyper-parameters}
Table of Hyper-parameters for both experiments. 


 

 

 

 


 





\end{document}
